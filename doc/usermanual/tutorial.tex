% ======================================================================
\section{Tutorial}
\label{s:tutorial}
% ======================================================================

If you encounter any problems that you cannot solve, please check in 
\url{https://twiki.cern.ch/twiki/bin/viewauth/CMS/Pxar#FAQ} for a
similar problem and solution, post a question to
\href{https://hypernews.cern.ch/HyperNews/CMS/get/pixel-psi46-testboard.html}{\tt
  pixel-psi46-testboard hypernews}, or open an issue at \url{https://github.com/psi46/pxar/issues/new}.

% ----------------------------------------------------------------------
\subsection{Installation}
\label{ss:installation}

You need the following packages/programs on your computer: 
\begin{itemize}
  \item C++ compiler
  \item ROOT 5.34/xx. Do not attempt to use ROOT 5.99, the
    beta-release of ROOT 6, as it will not work. 
  \item git
  \item cmake
  \item USB (header files)
  \item FTDI drivers
\end{itemize}
\fixme[add much more details. Volunteers?!]

% ----------------------------------------------------------------------
\subsection{Setup}
\label{ss:setup}
The \pxar code is hosted at \url{https://github.com/psi46/pxar}. Get it with 
{\tt
\begin{verbatim}
git clone git@github.com:psi46/pxar 
cd pxar
mkdir build
\end{verbatim}
}
If you run into problems here, you likely should setup git correctly
(see section~\ref{ss:installation}). 

For the time being, it is recommended that you get the HEAD of the
master branch. 


% ----------------------------------------------------------------------
\subsection{Compilation}
\label{ss:compilation}
Assuming that you start from the {\tt pxar} directory, do
{\tt
\begin{verbatim}
cd build
cmake -DBUILD_pxarui=ON ..
make [-j4] install [VERBOSE=1]
\begin{verbatim}
}
This will install by default the library into pxar/lib and the
executables into pxar/bin 
(both directories are subfolders of your local pxar folder).

Other cmake build options include
{\tt
\begin{verbatim}
cmake -DBUILD_pxarui=OFF ..
cmake -DBUILD_dummydtb=ON ..
\end{verbatim}
}

% ----------------------------------------------------------------------
\subsection{DTB firmware}
\label{ss:firmware}


To properly operated the DTB, you also need to have the matching
firmware loaded onto that device. The firmware can be obtained from 
\url{https://github.com/psi46/pixel-dtb-firmware/tree/master/FLASH} and the loaded
onto the DTB (its FPGA) with the following statements (assuming that
you are in directory {\tt pxar}): 
{\tt
\begin{verbatim}
cd ..
git clone git@github.com:psi46/pixel-dtb-firmware.git
cd pxar/main
../bin/pXar -f ../../pixel-dtb-firmware/FLASH/FLASHFILE
\end{verbatim}
} 
where {\tt FLASHFILE} should be replaced with the name of the most
recent version. The download will take a while. Follow the
instructions at the end: Wait until all 4 LEDs turn off and
power-cycle the DTB.

% ----------------------------------------------------------------------
\subsection{Program execution}
\label{ss:run}
The simplest way to run something of the \pxar framework is to use the
standalone test program: 
{\tt
\begin{verbatim}
../bin/testpxar
\end{verbatim}
}

To run the GUI, do 
{\tt
\begin{verbatim}
../bin/pXar -d ../data/defaultParametersRocPSI46digV2 -g [-v WARNING|QUIET|...]
../bin/pXar -d ../data/defaultParametersModulePSI46digV2 -g [-v WARNING|QUIET|...]
\end{verbatim}
}

and without the GUI, using a 'standard' ROOT C++ macro
{\tt
\begin{verbatim}
../bin/pXar -d ../data/defaultParametersRocPSI46digV2 \
  -c '../scripts/singleTest.C("DacScan", "DacScan.root")'
\end{verbatim}
}